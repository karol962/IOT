\documentclass{article}
\usepackage{graphicx}
\usepackage{color}
\usepackage{polski}
\usepackage[utf8]{inputenc}
\begin{document}

\title{Sprawozdanie lab1}
\author{Lesiak Karol}
\maketitle

\section{Wprowadzenie}

LaTeX – oprogramowanie do zautomatyzowanego składu tekstu, a także związany z nim język znaczników, służący do formatowania dokumentów tekstowych i tekstowo-graficznych (na przykład: broszur, artykułów, książek, plakatów, prezentacji, a nawet stron HTML)


\subsection{Tabela z wzorami, przykład formatowania tekstu}
\begin{tabular}{|c|c|p{6.5cm}|}
\hline
Lp. & \ \color{red} \textbf{Wzór} & \color{blue} \textit{Opis} \\ 
\hline
1 & $(a+b)^{2}=a^{2}+2ab+b^{2}$ & Wzór skróconego mnożenia \\
\hline
2 & $|AB|=\sqrt{(X_B - X_A)^2 + (Y_B - Y_A)^2}$ & Długość odcinka o końcach w punktach    \newline   $A=(x_A,y_A), B=(x_B,y_B)$\\
\hline
3 & $ E = mc^2  \hspace{0.5cm}   m = \frac{m_0}{\sqrt{1-\frac{v^2}{c^2}}}$  & E – energia, m - masa spoczynkowa ciała, c – prędkość światła \\
\hline
\end{tabular}

\bigskip
\underline{Opis wykorzystanego środowiska}
\bigskip


\colorbox{yellow}{Texmaker} to {\small{darmowy}}, {\large{nowoczesny}} i {\Large{wieloplatformowy}} {\LARGE{edytor}} {\huge{LaTeX}} dla systemów {\textrm{Linux}}, {\texttt{MacOSx}} i {\textsc{Windows}}, który integruje wiele narzędzi potrzebnych do tworzenia dokumentów za pomocą LaTeXa w jednej aplikacji.
\newline
\newline

\resizebox{13cm}{3cm}{IoT Fundamentals: Connecting Things}
\newline
\newline

\medskip

Wyrównanie 

\begin{flushright}
\textcolor{red}{Wyrównanie do prawej}
\end{flushright}
\begin{flushleft}
\textcolor{green}{Wyrównanie do lewej}
\end{flushleft}

\medskip

\section{Bibliografia}
\begin{thebibliography}{9}

\bibitem{a}https://pl.wikibooks.org/wiki/LaTeXZar%C4%85dzanie_bibliograf%C4%85/
  
\bibitem{b}$http://www.bibtex.org/Using/$

\bibitem{c}$http://www.fuw.edu.pl/~kostecki/kurs_latexa.pdf$

\bibitem{d}$http://www.latex-kurs.x25.pl$
\end{thebibliography}

\end{document}